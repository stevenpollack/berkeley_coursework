\documentclass[12pt,titlepage]{article}

\usepackage{mcgill,palatino,fancyhdr}

\lhead{STAT201A -- Sec. 102}
\chead{HW \#9 }
\rhead{Steven Pollack -- 24112977}
\cfoot{\thepage}

\title{STAT201A -- Sec. 102 \\ Homework \#9 }
\author{Steven Pollack \\ 24112977}
\date{}

\renewcommand{\P}{\mathbb{P}}

\begin{document}
\maketitle

\pagestyle{empty}
\newpage
\pagestyle{fancy}

\paragraph{\#1.}
\begin{proof}
Let $(I,II)$ be the random vector that represents the color of the ball drawn from box $I$ and box $II$. So, $(I,II) = (B,W)$ means a blue ball was drawn from $I$ and a white from $II$. Furthermore, set $X_n$ to be the count of white balls in box $I$, and note there exist three cases
\begin{align*}
X_{n} &= X_{n-1} -1 &\EQ (I,II) &= (W,B) \\
X_{n} &= X_{n-1} &\EQ (I,II) &= (B,B) \text{ or } (W,W) \\
X_{n} &= X_{n-1} + 1 &\EQ (I,II) &= (B,W) 
\end{align*}
Also, we note that $X_0 \sim $ HypGeo($N=10,G=4,n=5$). Now, it's clear that $P(X_n = i_n \given{ X_0 = i_0, X_1 = i_1, \ldots, X_{n-1} = i_{n-1}}) = P(X_n = i_n \given{X_{n-1}=i_{n-1}})$ thus $\set{X_{n}}$ is an irreducible, aperiodic Markov chain with a finite state-space. Hence, our big theorem says that there exists a unique distribution $\pi$ such that $\pi = \pi \P$. So, let's try calculating $\P$.

First, for $i,j \in \set{0,1,2,3,4}$, if $j \neq i, i\pm 1$, then $p_{i,j} = 0$. Next, from our analysis above, 
\begin{align*}
p_{i,i} &= P((I,II) = (B,B) \text{ or } (W,W) \given{\text{box I has $i$ white balls}}) &&= \frac{5-i}{5}\frac{1+i}{5} + \frac{i}{5}\frac{4-i}{5} \\
p_{i,i+1} &= P((I,II) = (B,W) \given{\text{box I has $i$ white balls}}) &&= \frac{5-i}{5}\frac{4-i}{5} \\
p_{i,i-1} &=  P((I,II) = (W,B) \given{\text{box I has $i$ white balls}}) &&= \frac{i}{5}\frac{1+i}{5}
\end{align*}
Which yields
\[
\P =\frac{1}{25}
\left(
\begin{array}{ccccc}
 5 & {20} & 0 & 0 & 0 \\
 2 & 11 & 12 & 0 & 0 \\
 0 & 6 & 13 & 6 & 0 \\
 0 & 0 & 12 & 11 & 2 \\
 0 & 0 & 0 & 20 & 5
\end{array}
\right)
\]
and if we let $\pi_{i} = P(X_0 = i) = \binom{4}{i}\binom{6}{5-i}/\binom{10}{5}$, then we see that $\pi = (\pi_0, \pi_1, \pi_2, \pi_3, \pi_4)$ solves the balance equations, $\pi \P = \pi$. 

Hence, $\pi$ is our stationary distribution and thus $\pi_0 = 1/42$ is the amount of time we expect box $I$ to be empty.
\end{proof}

\paragraph{\#2.}
\begin{proof}
\begin{enumerate}
\item Let $T_j$ be the random variable that counts the number of steps a chain $\set{X_n}$ takes to reach state $j$, and write $m_{ij} = E(T_j \given{X_0 = i})$. Then,
\begin{align*}
E(T_j \given{X_0=i}) &= \SUM{k}{1}{\infty} k P(T_j = k\given{X_0=i}) \\
&= P(T_j = 1\given{X_0 = i}) + 2 P(T_j =2 \given{X_0=i}) + 3 P(T_j = 3 \given{X_0 = i}) + \cdots \\% 4 P(T_j = 4 \given{X_0 = i}) + \cdots 
&= p_{ij} + 2\sum_{k_1 \neq j} p_{i k_1}p_{k_1 j} + 3 \sum_{k_1,k_2 \neq j} p_{i k_1}p_{k_1 k_2} p_{k_2 j} + \cdots \\
&= \left(p_{ij} + \sum_{k_1 \neq j} p_{i k_1}p_{k_1 j} +  \sum_{k_1,k_2 \neq j} p_{i k_1}p_{k_1 k_2} p_{k_2 j} + \cdots \right)  \\
&+ \left( \sum_{k_1 \neq j} p_{i k_1}p_{k_1 j} +  2 \sum_{k_1,k_2 \neq j} p_{i k_1}p_{k_1 k_2} p_{k_2 j} + 3 \sum_{k_1,k_2,k_3\neq j}p_{i k_1}p_{k_1 k_2} p_{k_2 k_3} p_{k_3 j} + \cdots \right) \\
&= S_1 + S_2
\end{align*}
However, notice that $S_1 = \SUM{k}{1}{\infty} P(T_j = k \given{X_0 = i})$, hence $S_1 = 1$. Furthermore, 
\begin{align*}
S_2 &= \sum_{k_1 \neq j} p_{i k_1}\left( p_{k_1 j} +  \sum_{k_2 \neq j} 2 p_{k_1 k_2} p_{k_2 j} + \sum_{k_2,k_3\neq j}3 p_{k_1 k_2} p_{k_2 k_3} p_{k_3 j} + \cdots \right) \\
&= \sum_{k_1 \neq j} p_{i k_1} E(T_j \given{X_0 = k_1}) \\
&= \sum_{k_1 \neq j} p_{i k_1} m_{k_1 j}
\end{align*}
Thus,
\[
m_{ij} = 1 + \sum_{k\neq j} p_{ik} m_{kj}
\]
\item Using the result in part 1.
\begin{align*} % E\left(T_{j}\given{X_0}\right)
\sum_{i} \pi_i m_{ij} &= \sum_{i}\left( \pi_{i} + \sum_{k\neq j} \pi_i p_{ik}m_{kj}\right) \\
\EQ E_{X_0}\left[m_{ij}\right]&= E_{X_0}\left[1 +\sum_{k\neq j} p_{ik}m_{kj}\right] \\
\EQ E_{X_0}\left[m_{ij}\right] &= 1 + \sum_{k\neq j}  E_{X_0}\left[p_{ik}m_{kj}\right] \\
\EQ  E_{X_0}\left[m_{ij}\right] &= 1 + \sum_{k\neq j} m_{kj} E_{X_0}\left[p_{ik}\right] \\
\EQ  E_{X_0}\left[m_{ij}\right] &= 1 + \sum_{k\neq j} m_{kj}\pi_k
\end{align*}
Where the second to last equality stems from the fact that $\sum_{k} \pi_{i}p_{ik} = \pi_k$ since $\pi$ is the stationary distribution of $\set{X_n}$. Finally, we observe that \[E_{X_0}[m_{ij}] = \SUM{k}{1}{n} m_{kj}\pi_k = m_{jj} \pi_{j} + \sum_{k\neq j} m_{kj} \pi_k\] Thus,
\[
 E_{X_0}\left[m_{ij}\right] = 1 + \sum_{k\neq j} m_{kj}\pi_k  \EQ m_{jj} \pi_j = 1 \EQ m_{jj} = \frac{1}{p_j}
\]

\end{enumerate}

\end{proof}


\paragraph{\#3.}
\begin{proof}
Set $D_1 = Y - \hat{Y}$, and $D_2 = X_2 - \hat{X}$, and by the hint, we note that $\cov(D_1, X_i) = 0$ for $i=1,2$ and $\cov(D_2, X_1) = 0$. Also, note that we may write $\hat{X} = X_2 - D_2$, hence
\begin{align*}
\tilde{Y} &= \mu_{Y} + c_{*1}(X_1 - \mu_{X_1}) + c_{*2}(\hat{X}-\mu_{X_2}) \\
&= \mu_{Y} + c_{*1}(X_1 - \mu_{X_1}) + c_{*2}(X_2 -\mu_{X_2} - D_2) \\
&= \hat{Y} - c_{*2} D_2
\end{align*}
Thus, using our first observations 
\begin{align*}
\cov(Y - \tilde{Y}, X_1) &= \cov((Y-\hat{Y}) + c_{*2}D_2, X_1) \\
&= \cov(D_1, X_1) + c_{*2}\cov(D_2,X_1) \\
&= 0
\end{align*}
which shows that $\tilde{Y}$ is the best linear predictor of $Y$ based on $X_1$. 
\end{proof}

\end{document}
